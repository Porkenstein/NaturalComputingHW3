% !TEX root = BioInspired.tex

\chapter{Immunocomputing - Text Chapter 6}

\section{Problem 9.1}

Problems that a turing machine cannot solve:
\begin{enumerate}
\item	Halting problem
\item	Empty tape acceptance problem
\item	Empty set acceptance problem
\item	Regular machine recognition problem
\end{enumerate}

\section{Problem 9.2}

Name four NP-Complete and four NP-Hard problems:

\subsection{NP-Complete}

\begin{enumerate}
\item	Knapsack Problem
\item	Traveling Salesman problem
\item	Vertex cover problem
\item	Hamiltonian Path problem
\end{enumerate}

Disclaimer: the problem stated to NAME the problems, not describe them

\subsection{NP-Hard}
\begin{enumerate}
\item	Flow shop scheduling
\item	K-minimum spanning tree
\item	Nurse scheduling problem
\item	Quadratic assignment problem
\end{enumerate}

Disclaimer: the problem stated to NAME the problems, not describe them

\section{Problem 9.3}

Maxam-Gilbert vs Sanger DNA sequencing methods

\begin{enumerate}

\item Maxam-Gilbert: 
The DNA strands are subject to a chemical treatment that fractures it in 4 points where different reactions occur (G, A+G, C, C+T).  They are then placed in a sequencing gel where each segment can be visible and the sequence can be inferred. 
\item Sanger: 

This method copies the DNA strand to be sequenced using four chemically altered bases.  Each of the bases stops when it encounters a specific letter associated with the DNA proteins.  After all of the copying is done there will be four strands each with one DNA letter on them.  They are then put backed together like a puzzle to give the sequence of the original strand of DNA

\item Contrast: 
While the Maxam-Gilbert method was used for a long time, it is very time consuming and prone to human error when examining the sequence gel.  The Sanger method is much more refined and can produce a more accurate result because of the way it is structured.  
\end{enumerate}